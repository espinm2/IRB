\documentclass[12pt]{article}

%\usepackage{fullpage,epsfig,wrapfig,url,palatino,color}
\usepackage{epsfig,wrapfig,url,palatino,color}

\pagestyle{plain}

\renewcommand{\baselinestretch}{1.}

\setlength{\topmargin}{0in}
\setlength{\evensidemargin}{0in}
\setlength{\oddsidemargin}{0in}
\setlength{\headheight}{0in}
\setlength{\headsep}{0in}
\setlength{\footskip}{0.2in}
\setlength{\textheight}{9in}
\setlength{\textwidth}{6.5in}

\renewcommand{\topfraction}{0.99}
\renewcommand{\bottomfraction}{0.99}
\renewcommand{\textfraction}{0.01}
\renewcommand{\floatpagefraction}{0.01}
\renewcommand{\dbltopfraction}{0.99}
\renewcommand{\dblfloatpagefraction}{0.01}

\begin{document}

\noindent
{\bf Title of Proposal:} Evaluation of the usability of a web based 
sketching interface for the iterative design of architectural daylighting \\
{\bf Researcher:}   Max Espinoza\\
{\bf Address:}  MRC 309A\\
{\bf Phone:} 860 918 5972\\ 
{\bf Research Advisor (for students):}  Barbara Cutler \\
{\bf Department:}  Computer Science \\
{\bf Is this proposal related to a sponsored project?}  Yes \\
{\bf If yes,  please indicate:}  \\
Existing Award: (Fund \# A12016), NSF, \\
Immersive Architectural Daylighting Design Experience \\

\noindent
All investigators, including faculty supervisors, on this project must
complete the self-study course on protection of human research
subjects. \\
{\bf Certification:  I/We have completed the course:} \\
Max Espinoza (CS PhD Student) 8/22/12 \\ % Update date of completion here <--------------------------- TODO
Barbara M Cutler 7/2/08, refresher 11/2/11

\paragraph{Objective:}
Our goal in this study is to evaluate the effectiveness and usability of 
our web based sketching interface for the creation of closed architectural 
geometries and daylighting analysis.

\paragraph{Methods:}
% The participants wil We will use a video camera with audio to
% record the game play and save digital files of the game state during
% play.  During the exercise, participants will be asked to speak aloud
% to each other and to the researcher about their observations of the
% system and the overall interaction.  After the game play is completed,
% each participant will be asked to fill out a paper and pencil
% questionnaire about the SAR system, the game interaction design and
% visualization, and its usefulness in providing feedback and
% implementation of the game rules and game play.  The system
% introduction will take approximately 10 minutes, game play will last
% approximately 30 minutes (5-10 minutes per round), and the written
% questionnaire will take about 20 minutes to complete.  The entire
% study (system and game introduction, game play, and questionnaire)
% will last approximately 1 hour.  We will ensure each participant is
% completed with the entire study in a maximum of 1.5 hours.
% 
% Our primary data collection device is the post-study paper
% questionnaire.  If the participants allow, we will also video tape the
% session.  The video tape is not necessary for completion of our study.
% Participants that allow us to video tape may be used as samples in our
% paper, and will allow us to do further analysis related to the timing
% of game actions, etc.  The data collected from video taping and audio
% taping will be fully anonymized before publication.  Faces and bodies
% will not be visible in the imagery.  And we will not publish the audio
% files, only possible extracted quotations.
% 
% The participant will be told that they are under no obligation to
% participate in the study, and that they may withdraw from the study at
% any point, without giving a reason.  
% 
% Participants for this study will be compensated for their time in the
% form of a gift certificate at the rate of \$10 per hour.  This
% compensation is not contingent upon the subject completing the entire
% study and will be prorated if the subject withdraws.  The
% participant's performance in the game (winning, losing, playing well
% or poorly) will not affect compensation.

\paragraph{Effects on Subjects:} See benefits and risks.

\paragraph{Benefits to Participants:}
The participants will gain first-hand experience with a new online sketching
environment for the generation of 3D models, and an understanding of the 
iterative design process of creating architectural spaces. The results 
of the study will lead to the use, development, and improvement of algorithms 
interpret architectural sketches. In addition to improvements in our daylighting 
visualizations.

\paragraph{Possible Risks:}   
% The spatially augmented reality system is identical to that from our
% earlier user study \#894 “Evaluation of the Virtual Heliodon for
% Architectural Daylighting Design.  The risks are the same for these
% two studies.  We will follow the same safety precautions and
% participant instructions as in that study.
% 
% The participants will be standing around a small table, manipulating
% small cardboard objects on the table and observing imagery projections
% on the table surface.  The table is surrounded by a heavy duty
% aluminum truss frame with 6 projectors mounted on the frame.
% 
% There is a risk of permanent eye damage if participants stand close
% to (30 centimeters or less) and within the beam of the projectors and
% look directly into the lens for more than 2 seconds.  The study does
% not require participants to position themselves or direct their gaze
% in such a way that this damage would occur.

\paragraph{Measures to Minimize Risk:}
\noindent

\paragraph{Likelihood of Physical Harm:}   None.

\paragraph{Alternate Method Not Using Human Subjects:} None

\paragraph{Qualifications of Researcher:}
Barbara Cutler has a PhD in Computer Science from Massachusetts
Institute of Technology.  Max Espinoza is a 3th year PhD student in
Computer Science at Rensselaer Polytechnic Institute studying computer
graphics.

\paragraph{Recruiting of Subjects:}
We will ask for online volunteers regardless of experience in architectural 
daylighting to use our web interface. We will advertise our web page to student 
volunteers, 18 years or older, via posters, campus related social media, 
and direct contact with students attending related courses. We will obtain 
permission of the course instructors to advertise for the participation of their
students, but participation in the study will be voluntary and will not impact 
their course grade.  The faculty advisor for the study (Barbara Cutler) will 
not recruit students in her courses to participate.  The names of the 
students who did or did not participate in the study will be confidential 
and will not be released to their instructors.

We are particularly interested in studying people with experience in 3D modeling, 
architectural daylighting, general architecture, and visual arts.
We are looking for users with formal education and experience in the above fields
because it provides us valuable feedback to improve our web interface usability
and functionality. This experience is not required for participation 
in the user study, but the subject's interest and prior experience in 
these related fields will be noted, and thus makes participation in 
this study simple and fun for the participant.  

% We will ask for student volunteers, 18 years or older, from the Games
% and Simulation Arts and Sciences courses and major in the School of
% Humanities, Arts, and Social Sciences.  We will obtain permission of
% the course instructors to advertise for the participation of their
% students, but participation in the study will be voluntary and will
% not impact their course grade.  The faculty advisor for the study
% (Barbara Cutler) will not recruit students in her courses to
% participate.  The names of the students who did or did not participate
% in the study will be confidential and will not be released to their
% instructors.
% 
% We are specifically interested in studying people with interest in the
% field of Games and Simulations Arts and Sciences (GSAS) and will
% advertise the study to GSAS majors.  Most or all students in this
% major have extensive prior experience using computer software,
% computer games, and graphical user interfaces.  Most or all students
% in this major are also very familiar with board games and many are
% familiar with specific ``miniature war games'', similar to the game
% used in our study.  This experience is not required for participation
% in the user study, but the subject's interest and prior experience
% with these interfaces and games is noted, and thus makes participation
% in this study simple and fun for the participant.  

\paragraph{Confidentiality:} % <--TODO Do we need to destroy the correspondence between models and user?
Participants will register themselves with a username of their choice,
when logging into the web interface for the first time. They will also provide
a password, which we will encrypt using \textbf{method goes here}. No where in 
our application will the user be asked for their real name. Participants 
will be identified by a randomly assigned ID number that is used only for 
this study. All recorded user information and design files will be labeled 
with this ID (and not the participant's username). All information and data 
relating to the user study will be protected to secure confidentiality. 
All electronic files will be stored on password protected computers in 
locked offices, which can be accessed only by the investigators of the user 
study. The correspondence between ID number and participant username will 
be recorded by Barbara Cutler and stored on a password protected computer, 
accessible only by the her. This correspondence will be destroyed once analysis 
of the data is complete, within 1 year after participation in the study.

\newpage

\section{Overview of the Table Top Spatially Augmented Reality System}

%\noindent
%The proposed user study investigates the effectiveness of the
%visualization and multi-player interaction of the ``ARmy'' game
%developed by Andrew Dolce~\cite{dolce} for his Computer Science
%Masters thesis (Figure~\ref{FIGURE_army_screenshots}). The ARmy
%application is a military simulation game played between two
%opponents. The game is in many ways similar to a typical tabletop war
%board game, in that it uses miniature objects to create a physical
%representation of the game world. The players are given a number of
%plastic soldier figurines or units, which represent their respective
%armies. Each player moves his units through the scene according to the
%rules of the game, engaging in combat with opposing units to eliminate
%them from play.

%\begin{figure}[h]
%\centering
%\resizebox{5in}{!}{\includegraphics{army_screenshots}} \\
%\caption{ Imagery from Andrew Dolce's ARmy military simulation game
%  played on our Table Top Spatially Augmented Reality System.}
%\label{FIGURE_army_screenshots}
%\end{figure}

The participants position a set of small boxes, ramps, and partition
walls made of lightweight white foamcore (foam+cardboard) and red or
green plastic army units within the workspace to construct the 3D
geometry of the game environment and control game play.  Images of the
environment are captured by a camera mounted above the scene and are
processed using computer vision to detect the current game state and
create a 3D digital model.  The ARmy computer application calculates
the legal movement range of each unit and also calculates the line of
sight and distance between opposing units to determine if combat will
occur between opposing units and if an advantage is afforded to one
unit over the other because of elevation differences.  The projectors are used to
directly augment the physical objects:

\end{document}
